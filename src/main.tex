\documentclass[a4paper, 10pt, oneside]{article}       % nechceme twoside
\usepackage{lmodern}                                  % latin modern font
\usepackage[utf8]{inputenc}                           % UTF-8
\usepackage[a4paper, left=20mm, right=30mm, top=30mm, bottom=35mm]{geometry}
\usepackage{booktabs}
\usepackage[table]{xcolor}
\usepackage{xcolor}
\usepackage{multirow}
\usepackage{siunitx}
\usepackage{listings}
\usepackage{caption}

\usepackage[ruled, vlined, czech] {algorithm2e}
\usepackage[czech=quotes]{csquotes}
\usepackage{cmap}
\usepackage{graphicx}
\usepackage[T1]{fontenc}
\usepackage[style=numeric,backend=biber,sorting=nty]{biblatex}
\usepackage[czech]{babel}                             % tenhle musi byt za biblatex

\usepackage{graphicx}
\usepackage[colorlinks=true, allcolors=blue]{hyperref}
\usepackage{verbatim}
\usepackage{algorithm2e}

\usepackage{amsmath}
\usepackage{amssymb}                                  % velice důležité pro symboly

\usepackage{tikz}
\usetikzlibrary{graphs,graphs.standard}

\newcolumntype{e}{D{,}{,}{3.5}}                       % zarovnání hodnot v tabulce podle .

% Úvodní strana
\usepackage{titling}
\title{\Huge\textbf{Diskrétní matematika}\\
{\textbf{Projekt}}\\
{\Large číslo zadání \underline{\makebox[1cm]{5}}}}
\author{}                                             % nutno ponechat prázdné
\date{}
\renewcommand\maketitlehooka{\null\mbox{}\vfill}      % vertically CENTER title page!!!
\renewcommand\maketitlehookd{\vfill\null}

\usepackage{fancyhdr}
\renewcommand{\headrulewidth}{0pt}                    % odstraní čáru pod textem v záhlaví

%\usepackage[useregional]{datetime2}                  % pro hod:min pomocí \DTMnow místo \today
\pagestyle{fancy} 				                      % nastavení stylu záhlaví a zápatí
\fancyhead[LO]{\Large VŠB - TUO}
\fancyhead[RO]{\Large Datum \underline{\the\day.\the\month.\the\year}}
\fancyfoot[LO]{\Large Osobní číslo \underline{\makebox[2.5cm]{STE0610 }}}
\fancyfoot[RO]{\Large Jméno \underline{\makebox[4cm]{Filip Štegner}}}
\fancyfoot[C]{}                                       % smaže číslování titulní strany!!!

% ZAČÁTEK*********************************************
\begin{document}

\shorthandoff{-}

% TITULNÍ STRANA
\maketitle\thispagestyle{fancy}

\vfill
\hfill
\begin{table}[b]
  \begin{tabular}{c|p{3cm}}
    \textbf{Příklad} & \textbf{Poznámky} \\
    \hline
        \rule{0pt}{1cm} \centering 1 &  \\
        \rule{0pt}{1.5cm}\\
        \rule{0pt}{1cm} \centering 2 &  \\
        \rule{0pt}{1.5cm}\\
  \end{tabular}
\end{table}

\clearpage                                            % zprovozní náš custom header!!!

% Záhlaví a zápatí
\fancyhead{}                                          % clear all header fields
\fancyfoot{}                                          % clear all footer fields
%\fancyhf{}                                           % clear existing header/footer entries

\fancyhead[LO]{\textit{Diskrétní matematika}}         % text v záhlaví

\fancyhead[RO]{\thepage}                              % číslování v záhlaví

\newpage
\setcounter{page}{1}
\section{Kombinatorika}\label{sec:sec_1}

\textit{V rozvodné síti přípojných bodů označených $0, 1, . . . , N$ sledujeme dostupný příkon $P_k$ v každém
bodě. Na vstupu do prvního bodu $(k = 0)$ je přiveden příkon $P_{in} = 230$. Regulace přenosu probíhá
po úsecích a má dvě složky: \newline
\begin{itemize}
  \item[-] \textbf{Tlumení skoků přenosu: } změnou příkonu mezi dvěma sousedními body označme $\varDelta_k := P_k - P_{k-1}$.
  Při přechodu na další úsek se tato změna přetlumí koeficientem $ r \in (0, 1) $ a obrátí vliv (regulace odbourává předchozí skok): 
  na dalším úseku tedy platí
  $$ \varDelta_{k+1} = -r \varDelta_k. $$
  \item[-] \textbf{Lokální odběr: } v bodě s indexem $k$ se odebere výkon $s_k$ rostoucí lineárně s indexem
  $$ s_k = s_0 + \alpha k $$
\end{itemize}
Skutečný přechod z $k$ na $k+1$ tedy kombinuje tlumení skoku i místní odběr:
$$ \varDelta_{k+1} = -r \varDelta_k - s_k, \varDelta_0 = 0, P_0 = P_{in} $$
\begin{enumerate}
  \item Z definice $\varDelta_k$ odvoďte vztah pro příkon v k-tem bodě závislý na příkonech ve dvou předchozích bodech a lokálním odběru platný pro $k = 2, . . . , N$.
  \item Určete typ rovnice a obecné řešení pro dostupný příkon v k-tem bodě, pokud víte, že $r = 1/2$, $s_0 = 2$, $\alpha = 6$.
  \item Pro $N = 5$, $r = 1/2$, $s_0 = 2$, $\alpha = 6$, $P_{in} = 230$ spočítejte $P_1,..., P_5$. Ověřte, že pro $k = 0, ..., 4$ platí $\varDelta_{k+1} = -r \varDelta_k - s_k$ a stručně okomentujte trend $P_k$.
\end{enumerate}
}
\subsection{Řešení:}

\textit{
S
}

\newpage
\section{Teorie grafů}\label{sec:sec_2}

\textit{Energetická společnost plánuje propojit osm výrobních uzlů \textrm{A, B, C, D, E, F, G, H}. Mezi některými
dvojicemi uzlů lze vybudovat přímé vedení. Každé vedení má váhu odpovídající nákladům (váha).
vedení jsou obousměrná, všechny váhy jsou kladné.
\begin{center}
\begin{table}[h]
  \centering
  \begin{tabular}{|c|c|c|c|cc}
    \multicolumn{1}{|l|}{Spojení} & \multicolumn{1}{l|}{Váha} & \multicolumn{1}{l|}{Spojení} & \multicolumn{1}{l|}{Váha} & \multicolumn{1}{l|}{Spojení} & \multicolumn{1}{l}{Váha} \\ \hline
    A-B                           & 4                         & B-C                          & 2                         & \multicolumn{1}{c|}{D-E}     & 3                        \\
    A-C                           & 5                         & B-D                          & 6                         & \multicolumn{1}{c|}{D-F}     & 5                        \\
    A-E                           & 9                         & C-E                          & 7                         & \multicolumn{1}{c|}{E-F}     & 4                        \\
    B-E                           & 3                         & C-F                          & 8                         & \multicolumn{1}{c|}{F-G}     & 2                        \\
    B-G                           & 9                         & D-G                          & 4                         & \multicolumn{1}{c|}{G-H}     & 3                        \\
    E-H                           & 6                         & F-H                          & 5                         & \multicolumn{2}{c}{---}                                
  \end{tabular}
\end{table}
\end{center}
\begin{enumerate}
  \item Nakreslete graf odpovídající uvedené tabulce propojení.
  \item Ověřte, že je graf souvislý.
  \item určete nejnižší náklady potřebné na vybudování jednotlivých vedení začínající v uzlu A a vedoucí do každého uzlu.
  \item Pokud bychom chtěli nalézt nejlevnější libovolné vedení propojující věchny uzly (ne nutně cestu), jaký jiný algoritmus bychom k tomu mohli použít?
\end{enumerate}
}

\subsection{Řešení:}
\begin{enumerate}
  \item \textit{Nakreslený graf:
  \begin{center}
  \begin{tikzpicture}[scale=1.2]
    \begin{scope}[every node/.style={circle,thick,draw}]
      \node (A) at (90:3.5) {A};
      \node (B) at (30:3.5) {B};
      \node (D) at (330:3.5) {D};
      \node (G) at (295:4) {G};
      \node (F) at (210:3.5) {F};
      \node (C) at (150:3.5) {C};
      \node (H) at (245:4) {H};
      \node (E) at (35:1) {E};
    \end{scope}
    \begin{scope}[every node/.style={fill=white, circle},
      every edge/.style={draw=black, thick}]
      \path [-] (A) edge node {$4$} (B);
      \path [-] (A) edge node {$5$} (C);
      \path [-] (B) edge node {$6$} (D);
      \path [-] (B) edge node {$3$} (E);
      \path [-] (C) edge node {$7$} (E);
      \path [-] (C) edge node {$8$} (F);
      \path [-] (D) edge node {$5$} (F);
      \path [-] (D) edge node {$4$} (G);
      \path [-] (E) edge[bend right=7] node {$6$} (H);
      \path [-] (F) edge node {$2$} (G);
      \path [-] (F) edge node {$5$} (H);
      \path [-] (G) edge node {$3$} (H);
      \path [-] (B) edge node {$2$} (C);
      \path [-] (A) edge[bend left=15] node {$9$} (E);
      \path [-] (D) edge[bend left=20] node {$3$} (E);
      \path [-] (E) edge node {$4$} (F);
      \path [-] (B) edge node {$9$} (G);
    \end{scope}
  \end{tikzpicture}
\end{center}
  }
\end{enumerate}
 
\end{document}